\chapter{The Zhang-Li Spin-transfer Torque Model}

Several approaches to deriving the spin-transfer torque exerted on local magnetic moments by spin-polarized conduction electrons have been proposed. The basic Hamiltonian for $s$-$d$ model may be written as \cite{Zhang2004}:
\begin{IEEEeqnarray}{rCl}
\mathcal{H}_{sd} & = & -J_{ex}\bm{s}\cdot\bm{S} \\
\frac{\bm{S}}{S} & = & -\frac{\bm{M}\left(\bm{r},t\right)}{M_{sat}} \\
\mathcal{H}_{sd} & = & \frac{SJ_{ex}}{M_{sat}}\bm{s}\cdot\bm{M}\left(\bm{r},t\right) \label{eq:def3}
\end{IEEEeqnarray}The conduction electrons may be generated by an DC electric field or a time-varying magnetic field. The DC electric field directly generates charge and spin currents in the ferromagnet. However, the time-varying magnetic field drive magnetization motion that induces nonequilibrium spin density via the ``$s$-$d$'' interaction. Then, the conduction electron spin operator satisfies the generalized continuity equation:\begin{IEEEeqnarray}{rCl}
\frac{\partial{}\bm{s}}{\partial{}t} + \bm{\nabla}\cdot\bm{J} & = &\frac{1}{i\hbar}\left[\bm{s},\mathcal{H}_{sd}\right] - \Gamma_{re}(\bm{s}) \label{eq:contEq}
\end{IEEEeqnarray}where $\bm{J}$ is the spin current operator, and $\Gamma_{re}(\bm{s})$ represents the spin relaxation due to scattering impurities, electrons, etc. Using $<>$ for the average over all occupies electronic states, $\bm{m}(\bm{r},t)=<\bm{s}>$ is the electron spin density and $\mathcal{J}(\bm{r},t)=<\bm{J}>$ is the spin current density. Then, the semiclassical Bloch equation for the $\bm{m}$ is obtained:\begin{IEEEeqnarray}{rCl}
\frac{\partial{}\bm{m}}{\partial{}t}+\bm{\nabla}\cdot\mathcal{J} & = & -\frac{1}{\tau_{ex}M_{sat}}\bm{m}\times\bm{M}\left(\bm{r},t\right)-<\Gamma(\bm{s})> \label{eq:blochSpinDen}
\end{IEEEeqnarray}where the commutator in Eq.~(\ref{eq:contEq}) has been explicitly calculated using Eq.~(\ref{eq:def3}) , and $\tau_{ex}=\hbar/SJ_{ex}$. The induced spin density, $\bm{m}$, may then be separated into: \begin{IEEEeqnarray}{rCl}
\bm{m}\left(\bm{r},t\right) & = & \bm{m}_{0}\left(\bm{r},t\right) + \delta\bm{m}\left(\bm{r},t\right) = n_{0}\frac{\bm{M}\left(\bm{r},t\right)}{M_{sat}} + \delta\bm{m}\left(\bm{r},t\right) \label{eq:indSpinDen}
\end{IEEEeqnarray}where $n_{0}$ is the local equilibrium spin density whose direction is parallel to the magnetization. The first term in Eq.~(\ref{eq:indSpinDen}) is the adiabatic spin density when the conduction electron spin relaxes to its equilibrium value at an instantaneous time $t$. Here, it is assumed that dynamics of magnetization is slow compared to that of conduction electrons and hence, the spin of the conduction electrons approximately follows the direction of the local moment (known as the adiabatic process). The second term in Eq.~(\ref{eq:indSpinDen}) is the deviation from the adiabatic process.

As it turns out, the spin current density may also be written in a similar way:\begin{IEEEeqnarray}{rCl}
\mathcal{J}\left(\bm{r},t\right) & = & \mathcal{J}_{0}\left(\bm{r},t\right) + \delta\mathcal{J}\left(\bm{r},t\right) = -\frac{\mu_{B}P}{e}\bm{j_{e}}\otimes\frac{\bm{M}\left(\bm{r},t\right)}{M_{sat}} + \delta\mathcal{J}\left(\bm{r},t\right) \label{eq:spinCurr0}
\end{IEEEeqnarray}where $e$ is the electron charge, $j_{e}$ is the current density, $\mu_{B}$ is the Bohr magneton, and $P$ is the spin current polarization of the ferromagnet. The spin current is a tensor consisting of two vectors: the charge current and the spin polarization of the current ($\otimes$ denote tensor product). The first term in Eq.~(\ref{eq:spinCurr0}) is the spin current whose spin polarization is parallel to the local magnetization $\bm{M}\left(\bm{r},t\right)$.

The following simplifications are then used to obtain a closed form for the nonequilibrium spin density. First, a simple relaxation time approximation is used to model the relaxation term in Eq.~(\ref{eq:blochSpinDen}): $<\Gamma(\bm{s})>=\delta\bm{m}\left(\bm{r},t\right)/\tau_{sf}$ where $\tau_{sf}$ is the spin-flip relaxation time. Second, only the linear response of $\delta\bm{m}$ to the electric current $j_{e}$ and to $\partial\bm{M}/\partial{}t$. Since $\delta\bm{m}$ is already the first order, $\partial{}\delta\bm{m}/\partial{}t$ will be the order of $j_{e}\partial\bm{M}/\partial{}t$ or $\partial^{2}\bm{M}/\partial{}t^{2}$ and thus it can be discarded. In the semiclassical picture, the nonadiabatic current density $\delta\mathcal{J}$ is related to the nonequilibrium spin density $\delta\bm{m}$ as $\delta\mathcal{J}=-D_{0}\nabla\delta\bm{m}$ where $D_{0}$ is the diffusion constant. Substituting Eqs.~(\ref{eq:indSpinDen}) and (\ref{eq:spinCurr0}) into (\ref{eq:blochSpinDen}) and using the simplifications, we obtain:\begin{IEEEeqnarray}{rCl}
D_{0}\nabla^{2}\delta\bm{m}-\frac{1}{\tau_{ex}M_{sat}}\delta\bm{m}\times\bm{M} - \frac{\delta\bm{m}}{\tau_{sf}} & = & \frac{n_{0}}{M_{sat}}\cdot\frac{\partial\bm{M}}{\partial{}t}-\frac{\mu_{B}P}{eM_{sat}}(\bm{j_{e}}\cdot\nabla)\bm{M} \label{eq:nonEqSpinDen}
\end{IEEEeqnarray}The right-hand side of Eq.~(\ref{eq:nonEqSpinDen} shows that the source of nonequilibrium spin density is 1) the time variation and 2) spatial variation of $\bm{M}$.

Eq.~(\ref{eq:nonEqSpinDen}) may be written in the form of:\begin{IEEEeqnarray}{rCl}
\nabla^{2}y-\frac{1}{\lambda^{2}}y & = & f(\bm{r}) \label{eq:ode1}
\end{IEEEeqnarray}where $y=\delta{}m_{x}\pm{}i\delta{}m_{y}$ is the spin density rotating frame, $\delta{}m_{x}$ and $\delta{}m_{y}$ are two transverse components of $\delta\bm{m}$ perpendicular to $\bm{M}\left(\bm{r},t\right)$, and $\lambda=\sqrt{D_{0}(1/\tau_{sf}+i/\tau_{ex})^{-1}}$. The solution to Eq.~(\ref{eq:ode1}) has the form:\begin{IEEEeqnarray}{rCl}
y(\bm{r}) & = & \int\frac{exp\left(-|\bm{r}-\bm{r'}|/\lambda\right)}{4\pi|\bm{r}-\bm{r'}|}f\left(\bm{r'}\right)d^{3}\bm{r'}
\end{IEEEeqnarray}For slowly varying $f\left(\bm{r'}\right)$, we can replace $f\left(\bm{r'}\right)$ with $f\left(\bm{r}\right)$ in the integral and thus $y=-\lambda^{2}f\left(\bm{r}\right)$. Hence, $D_{0}\nabla^{2}\delta\bm{m}$ in Eq.~(\ref{eq:nonEqSpinDen}) may be discarded. Then, Eq.~(\ref{eq:nonEqSpinDen}) becomes:\begin{IEEEeqnarray}{rCl}
-\frac{1}{\tau_{ex}M_{sat}}\delta\bm{m}\times\bm{M} - \frac{\delta\bm{m}}{\tau_{sf}} & = & \frac{n_{0}}{M_{sat}}\cdot\frac{\partial\bm{M}}{\partial{}t}-\frac{\mu_{B}P}{eM_{sat}}(\bm{j_{e}}\cdot\nabla)\bm{M} \\
\frac{\delta\bm{m}}{\tau_{sf}} & = & -\frac{1}{\tau_{ex}M_{sat}}\delta\bm{m}\times\bm{M} - \frac{n_{0}}{M_{sat}}\cdot\frac{\partial\bm{M}}{\partial{}t} + \frac{\mu_{B}P}{eM_{sat}}(\bm{j_{e}}\cdot\nabla)\bm{M}
\end{IEEEeqnarray}Following the method for LLG, we may then write the last equation as:\begin{IEEEeqnarray}{rCl}
\frac{\delta\bm{m}}{\tau_{sf}} & = & \begin{IEEEeqnarraybox}[][c]{l}
-\frac{1}{\tau_{ex}{}M_{sat}}\left(-\frac{\tau_{sf}}{\tau_{ex}M_{sat}}\delta\bm{m}\times\bm{M} - \frac{n_{0}\tau_{sf}}{M_{sat}}\cdot\frac{\partial\bm{M}}{\partial{}t} + \frac{\mu_{B}P\tau_{sf}}{eM_{sat}}(\bm{j_{e}}\cdot\nabla)\bm{M}\right)\times\bm{M} \\
- \frac{n_{0}}{M_{sat}}\cdot\frac{\partial\bm{M}}{\partial{}t} + \frac{\mu_{B}P}{eM_{sat}}(\bm{j_{e}}\cdot\nabla)\bm{M}
\end{IEEEeqnarraybox} \\
& = & \begin{IEEEeqnarraybox}[][c]{l}
-\frac{1}{\tau_{ex}M_{sat}}\left(\frac{\tau_{sf}}{\tau_{ex}M_{sat}}\bm{M}\times\delta\bm{m}\times\bm{M} + \frac{n_{0}\tau_{sf}}{M_{sat}}\bm{M}\times\frac{\partial{}\bm{M}}{\partial{}t} - \frac{\mu_{B}P\tau_{sf}}{eM_{sat}}\bm{M}\times\left(\bm{j_{e}}\cdot\nabla\right)\bm{M}\right) \\
- \frac{n_{0}}{M_{sat}}\cdot\frac{\partial\bm{M}}{\partial{}t} + \frac{\mu_{B}P}{eM_{sat}}(\bm{j_{e}}\cdot\nabla)\bm{M}
\end{IEEEeqnarraybox} \\
\frac{\delta\bm{m}}{\tau^{2}_{sf}} & = & \begin{IEEEeqnarraybox}[][c]{l}
-\frac{1}{\tau_{ex}M_{sat}}\left(\frac{1}{\tau_{ex}M_{sat}}\bm{M}\times\delta\bm{m}\times\bm{M} + \frac{n_{0}}{M_{sat}}\bm{M}\times\frac{\partial{}\bm{M}}{\partial{}t} - \frac{\mu_{B}P}{eM_{sat}}\bm{M}\times\left(\bm{j_{e}}\cdot\nabla\right)\bm{M}\right) \\
- \frac{n_{0}}{\tau_{sf}M_{sat}}\cdot\frac{\partial\bm{M}}{\partial{}t} + \frac{\mu_{B}P}{\tau_{sf}eM_{sat}}(\bm{j_{e}}\cdot\nabla)\bm{M}
\end{IEEEeqnarraybox} \\
\frac{\tau^{2}_{ex}}{\tau^{2}_{sf}}\delta\bm{m} & = & \begin{IEEEeqnarraybox}[][c]{l}
-\frac{1}{M_{sat}}\left(\frac{1}{M_{sat}}\bm{M}\times\delta\bm{m}\times\bm{M} + \frac{n_{0}\tau_{ex}}{M_{sat}}\bm{M}\times\frac{\partial{}\bm{M}}{\partial{}t} - \frac{\mu_{B}P\tau_{ex}}{eM_{sat}}\bm{M}\times\left(\bm{j_{e}}\cdot\nabla\right)\bm{M}\right) \\
- \frac{n_{0}\tau^{2}_{ex}}{\tau_{sf}M_{sat}}\cdot\frac{\partial\bm{M}}{\partial{}t} + \frac{\mu_{B}P\tau^{2}_{ex}}{\tau_{sf}eM_{sat}}(\bm{j_{e}}\cdot\nabla)\bm{M}
\end{IEEEeqnarraybox} \\
\xi^{2}\delta\bm{m} & = & \begin{IEEEeqnarraybox}[][c]{l}
\frac{\bm{M}}{M_{sat}}\times\frac{\bm{M}}{M_{sat}}\times\delta\bm{m} - \frac{n_{0}\tau_{ex}}{M^{2}_{sat}}\bm{M}\times\frac{\partial{}\bm{M}}{\partial{}t} - \frac{n_{0}\tau_{ex}\xi}{M_{sat}}\frac{\partial\bm{M}}{\partial{}t}  \\
+ \frac{\mu_{B}P\tau_{ex}}{eM^{2}_{sat}}\bm{M}\times\left(\bm{j_{e}}\cdot\nabla\right)\bm{M} + \frac{\mu_{B}P\xi\tau_{ex}}{eM_{sat}}(\bm{j_{e}}\cdot\nabla)\bm{M}
\end{IEEEeqnarraybox}
\end{IEEEeqnarray}where we have used $\xi=\frac{\tau_{ex}}{\tau_{sf}}$. With more manipulations, we obtain:\begin{IEEEeqnarray}{rCl}
\xi^{2}\delta\bm{m} - \frac{\bm{M}}{M_{sat}}\times\frac{\bm{M}}{M_{sat}}\times\delta\bm{m} & = & \begin{IEEEeqnarraybox}[][c]{l}
\tau_{ex}\left(- \frac{n_{0}}{M^{2}_{sat}}\bm{M}\times\frac{\partial{}\bm{M}}{\partial{}t} - \frac{n_{0}\xi}{M_{sat}}\frac{\partial\bm{M}}{\partial{}t} \right. \\
\left. + \frac{\mu_{B}P}{eM^{2}_{sat}}\bm{M}\times\left(\bm{j_{e}}\cdot\nabla\right)\bm{M} + \frac{\mu_{B}P\xi}{eM_{sat}}(\bm{j_{e}}\cdot\nabla)\bm{M} \right)
\end{IEEEeqnarraybox} \label{eq:lastStep}
\end{IEEEeqnarray}If we assume that $\delta\bm{m}$ is orthogonal to $\bm{M}$, then the double cross product in Eq.~(\ref{eq:lastStep}) is just in the exact opposite direction of $\delta\bm{m}$, and:\begin{IEEEeqnarray}{rCl}
(1+\xi^{2})\delta\bm{m} & = & \tau_{ex}\left(- \frac{n_{0}}{M^{2}_{sat}}\bm{M}\times\frac{\partial{}\bm{M}}{\partial{}t} - \frac{n_{0}\xi}{M_{sat}}\frac{\partial\bm{M}}{\partial{}t} + \frac{\mu_{B}P}{eM^{2}_{sat}}\bm{M}\times\left(\bm{j_{e}}\cdot\nabla\right)\bm{M} + \frac{\mu_{B}P\xi}{eM_{sat}}(\bm{j_{e}}\cdot\nabla)\bm{M} \right) \\
\delta\bm{m} & = & \begin{IEEEeqnarraybox}[][c]{l}
\frac{\tau_{ex}}{(1+\xi^{2})}\left(- \frac{n_{0}\xi}{M_{sat}}\frac{\partial\bm{M}}{\partial{}t} - \frac{n_{0}}{M^{2}_{sat}}\bm{M}\times\frac{\partial{}\bm{M}}{\partial{}t} \right. \\
\left. + \frac{\mu_{B}P\xi}{eM_{sat}}(\bm{j_{e}}\cdot\nabla)\bm{M} + \frac{\mu_{B}P}{eM^{2}_{sat}}\bm{M}\times\left(\bm{j_{e}}\cdot\nabla\right)\bm{M} \right)
\end{IEEEeqnarraybox} \label{eq:delm}
\end{IEEEeqnarray}

According to \cite{Zhang2004}, the torque exerted is $\bm{T}=-(SJ_{ex}/\hbar{}M_{sat})\bm{M}\times\bm{m}=-(1/\tau_{ex}M_{sat})\bm{M}\times\delta\bm{m}$. Using Eq.~(\ref{eq:delm}),\begin{IEEEeqnarray}{rCl}
\bm{T} & = & \begin{IEEEeqnarraybox}[][c]{l}
\frac{1}{1+\xi^{2}}\left[- \frac{n_{0}}{M_{sat}}\frac{\partial\bm{M}}{\partial{}t} + \frac{n_{0}\xi}{M^{2}_{sat}}\bm{M}\times\frac{\partial\bm{M}}{\partial{}t} \right. \\
\left. - \frac{\mu_{B}P}{eM^{3}_{sat}}\bm{M}\times\bm{M}\times\left(\bm{j_{e}}\cdot\nabla\right)\bm{M} - \frac{\mu_{B}P\xi}{eM^{2}_{sat}}\bm{M}\times(\bm{j_{e}}\cdot\nabla)\bm{M}\right]
\end{IEEEeqnarraybox} \label{eq:ZLiTorque0}
\end{IEEEeqnarray}This torque is then entered into the Landau-Lifshitz-Gilbert (LLG) equation:\begin{IEEEeqnarray}{rCl}
\frac{\partial{}\bm{M}}{\partial{}t} & = & -|\gamma| \bm{M}\times\bm{H_{EFF}} + \frac{\alpha}{M_{sat}}\bm{M}\times\frac{\partial{}\bm{M}}{\partial{}t} + \bm{T}
\end{IEEEeqnarray}In \cite{Zhang2004}, the first two terms on the right-hand side of Eq.~(\ref{eq:ZLiTorque0}) are neglected because they renormalize the precession and damping terms in the LLG equation, respectively. Hence, only the last two terms are considered. This yields:\begin{IEEEeqnarray}{rCl}
\frac{\partial{}\bm{M}}{\partial{}t} & = & \begin{IEEEeqnarraybox}[][c]{l}-|\gamma| \bm{M}\times\bm{H_{EFF}} + \frac{\alpha}{M_{sat}}\bm{M}\times\frac{\partial{}\bm{M}}{\partial{}t} \\
- \frac{\mu_{B}P}{eM_{sat}(1+\xi^{2})}\left[\frac{\bm{M}}{M_{sat}}\times\frac{\bm{M}}{M_{sat}}\times\left(\bm{j_{e}}\cdot\nabla\right)\bm{M} + \xi\frac{\bm{M}}{M_{sat}}\times(\bm{j_{e}}\cdot\nabla)\bm{M}\right]
\end{IEEEeqnarraybox}
\end{IEEEeqnarray}Finally, we define $b_{J}=\mu_{B}Pj_{e}/eM_{sat}(1+\xi^{2})$, and $c_{J}=\mu_{B}Pj_{e}\xi/eM_{sat}(1+\xi^{2})=\xi{}b_{J}$, we obtain:\begin{IEEEeqnarray}{rCl}
\frac{\partial{}\bm{M}}{\partial{}t} & = & -|\gamma| \bm{M}\times\bm{H_{EFF}} + \frac{\alpha}{M_{sat}}\bm{M}\times\frac{\partial{}\bm{M}}{\partial{}t} - \frac{b_{J}}{M^{2}_{sat}}\bm{M}\times\bm{M}\times\left(\bm{j_{e}}\cdot\nabla\right)\bm{M} - \frac{c_{J}}{M_{sat}}\bm{M}\times(\bm{j_{e}}\cdot\nabla)\bm{M}
\end{IEEEeqnarray}